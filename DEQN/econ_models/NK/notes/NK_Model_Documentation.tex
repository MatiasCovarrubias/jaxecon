\documentclass[11pt,a4paper]{article}

% Packages
\usepackage[utf8]{inputenc}
\usepackage[margin=1in]{geometry}
\usepackage{amsmath,amssymb,amsthm}
\usepackage{graphicx}
\usepackage{hyperref}
\usepackage{booktabs}
\usepackage{xcolor}
\usepackage{listings}
\usepackage{enumitem}
\usepackage{titlesec}

% Colors
\definecolor{nkblue}{RGB}{46,134,171}
\definecolor{codegray}{RGB}{240,240,240}

% Hyperref setup
\hypersetup{
    colorlinks=true,
    linkcolor=nkblue,
    filecolor=nkblue,
    urlcolor=nkblue,
    citecolor=nkblue
}

% Section formatting
\titleformat{\section}
  {\normalfont\Large\bfseries\color{nkblue}}{\thesection}{1em}{}
\titleformat{\subsection}
  {\normalfont\large\bfseries\color{nkblue}}{\thesubsection}{1em}{}

% Code listing style
\lstset{
    backgroundcolor=\color{codegray},
    basicstyle=\ttfamily\small,
    breaklines=true,
    frame=single,
    xleftmargin=0.5cm,
    xrightmargin=0.5cm
}

% Title information
\title{\textbf{\huge Three-Equation New Keynesian Model}\\[0.5em]
       \Large Implementation in JAXecon/DEQN\\[0.3em]
       \large Based on Galí (2015), Chapter 3}
\author{}
\date{\today}

\begin{document}

\maketitle

\begin{abstract}
\noindent This document provides a comprehensive explanation of the three-equation New Keynesian (NK) model as presented in Chapter 3 of Galí's \textit{Monetary Policy, Inflation, and the Business Cycle} (2015), along with its implementation in the JAXecon/DEQN framework. The NK model is the canonical framework for monetary policy analysis, combining optimizing behavior with nominal rigidities to generate a meaningful role for monetary policy in stabilizing the economy.
\end{abstract}

\tableofcontents
\newpage

\section{Introduction}

The three-equation New Keynesian model is the workhorse framework for monetary policy analysis in modern macroeconomics. It represents a significant departure from classical and Real Business Cycle models by incorporating nominal price rigidities, which create a role for monetary policy to affect real economic variables in the short run.

\subsection{Key Features}

\begin{itemize}[leftmargin=*]
    \item \textbf{Micro-founded:} Derived from optimizing behavior of households and firms
    \item \textbf{Forward-looking:} All agents form rational expectations about the future
    \item \textbf{Nominal rigidities:} Prices are sticky (Calvo pricing)
    \item \textbf{No endogenous state variables:} Unlike RBC models, no capital accumulation
    \item \textbf{Divine coincidence:} In the baseline model, stabilizing inflation also stabilizes the output gap
\end{itemize}

\subsection{Structure}

The model consists of three equations:
\begin{enumerate}[leftmargin=*]
    \item \textbf{New Keynesian Phillips Curve (NKPC):} Links inflation to expected future inflation and the output gap
    \item \textbf{Dynamic IS Equation:} Relates the output gap to expected future output gap and the real interest rate
    \item \textbf{Monetary Policy Rule:} Specifies how the central bank sets the nominal interest rate (Taylor rule)
\end{enumerate}

The model is closed by specifying the exogenous shock processes: productivity shocks (affecting the natural rate of interest) and monetary policy shocks.

\newpage

\section{The Model Equations}

\subsection{New Keynesian Phillips Curve (NKPC)}

The NKPC is derived from Calvo pricing and links current inflation to expected future inflation and the output gap:

\begin{equation}
\boxed{\pi_t = \beta \, \mathbb{E}_t\{\pi_{t+1}\} + \kappa \, \tilde{y}_t}
\tag{Galí 21}
\end{equation}

\noindent where:
\begin{itemize}[leftmargin=2cm]
    \item[$\pi_t$:] Inflation rate (log deviation from steady state)
    \item[$\tilde{y}_t$:] Output gap (log deviation of output from its natural level)
    \item[$\beta$:] Household discount factor ($\beta = 0.99$ quarterly)
    \item[$\kappa$:] Slope of the Phillips curve
\end{itemize}

\vspace{0.3cm}
\noindent\textbf{Interpretation:} The NKPC embodies the supply side of the economy. Current inflation depends on expected future inflation (forward-looking pricing) and the output gap (marginal cost pressure). The parameter $\kappa$ determines the sensitivity of inflation to the output gap and is given by:
$$\kappa = \lambda\left(\sigma + \frac{\varphi + \alpha}{1-\alpha}\right)$$
where $\lambda$ depends on the degree of price stickiness (Calvo parameter).

\subsection{Dynamic IS Equation}

The Dynamic IS equation is the log-linearized household Euler equation:

\begin{equation}
\boxed{\tilde{y}_t = \mathbb{E}_t\{\tilde{y}_{t+1}\} - \frac{1}{\sigma}\left(i_t - \mathbb{E}_t\{\pi_{t+1}\} - r_t^n\right)}
\tag{Galí 22}
\end{equation}

\noindent where:
\begin{itemize}[leftmargin=2cm]
    \item[$i_t$:] Nominal interest rate (policy instrument)
    \item[$r_t^n$:] Natural rate of interest (efficient real rate under flexible prices)
    \item[$\sigma$:] Inverse elasticity of intertemporal substitution (CRRA coefficient)
\end{itemize}

\vspace{0.3cm}
\noindent\textbf{Interpretation:} The DIS represents the demand side. Current output gap depends on expected future output gap (consumption smoothing) and the deviation of the real interest rate ($i_t - \mathbb{E}_t\{\pi_{t+1}\}$) from the natural rate. When the real rate exceeds the natural rate, current demand falls relative to potential.

\subsection{Taylor Rule (Monetary Policy)}

The central bank sets the nominal interest rate according to a Taylor rule:

\begin{equation}
\boxed{i_t = \rho + \phi_\pi \, \pi_t + \phi_y \, \tilde{y}_t + v_t}
\tag{Galí 25}
\end{equation}

\noindent where:
\begin{itemize}[leftmargin=2cm]
    \item[$\rho$:] Steady-state real interest rate ($\rho = -\log(\beta)$)
    \item[$\phi_\pi$:] Response coefficient to inflation (typically $\phi_\pi > 1$ for determinacy)
    \item[$\phi_y$:] Response coefficient to output gap
    \item[$v_t$:] Monetary policy shock (exogenous deviation from rule)
\end{itemize}

\vspace{0.3cm}
\noindent\textbf{Taylor Principle:} For the equilibrium to be determinate (unique), the central bank must respond more than one-for-one to inflation: $\phi_\pi > 1$. This ensures that when inflation rises, the central bank raises the nominal rate by enough to increase the real interest rate, which dampens demand and brings inflation back down.

\newpage

\section{Natural Rate and Shock Processes}

\subsection{Natural Rate of Interest}

The natural rate of interest $r_t^n$ is the real interest rate that would prevail under flexible prices. It is determined by fundamentals (productivity):

\begin{equation}
\boxed{r_t^n = \rho + \sigma \, \psi_{ya}^n \, \mathbb{E}_t\{\Delta a_{t+1}\}}
\tag{Galí 23}
\end{equation}

\noindent where $a_t$ is the productivity shock and $\psi_{ya}^n$ is the elasticity of natural output to productivity.

For an AR(1) productivity process:
$$a_t = \rho_a \, a_{t-1} + \sigma_a \, \varepsilon_t^a, \quad \varepsilon_t^a \sim \mathcal{N}(0,1)$$

we have:
$$\mathbb{E}_t\{\Delta a_{t+1}\} = \mathbb{E}_t\{a_{t+1} - a_t\} = (\rho_a - 1) \, a_t$$

Therefore, in deviations from steady state:
\begin{equation}
r_t^n - \rho = \sigma \, \psi_{ya}^n \, (\rho_a - 1) \, a_t
\end{equation}

Since $\rho_a < 1$, a positive productivity shock \textit{lowers} the natural rate of interest.

\subsection{Exogenous Shock Processes}

The model features two exogenous AR(1) shock processes:

\paragraph{Productivity Shock} (affects natural rate):
$$a_t = \rho_a \, a_{t-1} + \sigma_a \, \varepsilon_t^a, \quad \varepsilon_t^a \sim \mathcal{N}(0,1)$$

\paragraph{Monetary Policy Shock} (deviation from Taylor rule):
$$v_t = \rho_v \, v_{t-1} + \sigma_v \, \varepsilon_t^v, \quad \varepsilon_t^v \sim \mathcal{N}(0,1)$$

\subsection{State Space Representation}

The model can be represented in state-space form:

\begin{align*}
\text{State variables (exogenous):} \quad & s_t = [a_t, \, v_t]^\top \\
\text{Policy variables (endogenous):} \quad & p_t = [\tilde{y}_t, \, \pi_t]^\top
\end{align*}

The equilibrium is characterized by a policy function $p_t = g(s_t)$ that satisfies the two equilibrium conditions (DIS and NKPC) at all points in the state space.

\newpage

\section{Calibration}

Table~\ref{tab:calibration} presents the default calibration for quarterly frequency, following Galí (2015).

\begin{table}[h]
\centering
\caption{Default Parameter Calibration (Quarterly)}
\label{tab:calibration}
\begin{tabular}{@{}llcl@{}}
\toprule
\textbf{Parameter} & \textbf{Symbol} & \textbf{Value} & \textbf{Description} \\
\midrule
Discount factor & $\beta$ & 0.99 & Implies 4\% annual real rate \\
CRRA coefficient & $\sigma$ & 1.0 & Log utility \\
NKPC slope & $\kappa$ & 0.1275 & Depends on price stickiness \\
Taylor: inflation & $\phi_\pi$ & 1.5 & Satisfies Taylor principle \\
Taylor: output gap & $\phi_y$ & 0.125 & = 0.5/4 (quarterly) \\
\midrule
Productivity persist. & $\rho_a$ & 0.9 & Persistent shock \\
Monetary persist. & $\rho_v$ & 0.5 & Less persistent \\
Productivity std. & $\sigma_a$ & 0.01 & 1\% std deviation \\
Monetary std. & $\sigma_v$ & 0.0025 & 25 basis points \\
\midrule
Natural output elast. & $\psi_{ya}^n$ & 1.0 & Productivity elasticity \\
\bottomrule
\end{tabular}
\end{table}

\section{Implementation in JAXecon/DEQN}

\subsection{File Structure}

The NK model is implemented in the following directory structure:

\begin{lstlisting}[language=bash]
DEQN/econ_models/NK/
├── __init__.py           # Module exports
├── model.py              # Model class (equilibrium conditions)
├── train.py              # Training script
├── analysis.py           # Impulse response analysis
├── create_documentation.py  # Documentation generator
└── notes/
    └── NK_Model_Documentation.pdf
\end{lstlisting}

\subsection{Model Class (\texttt{model.py})}

The core model is implemented in the \texttt{Model} class with the following key methods:

\subsubsection{\texttt{\_\_init\_\_(...)}}
Initializes model parameters and computes steady state values. Stores all parameters ($\beta$, $\sigma$, $\kappa$, $\phi_\pi$, $\phi_y$, etc.) and computes derived quantities like the natural rate coefficient.

\subsubsection{\texttt{step(state, policy, shock)}}
Implements the state transition function:
$$s_{t+1} = f(s_t, \varepsilon_{t+1}) = \begin{bmatrix} \rho_a \, a_t + \sigma_a \, \varepsilon_t^a \\ \rho_v \, v_t + \sigma_v \, \varepsilon_t^v \end{bmatrix}$$

Note that policies don't affect state transitions in the NK model (unlike RBC models with capital accumulation).

\subsubsection{\texttt{expect\_realization(state\_next, policy\_next)}}
Returns the realization of expectational terms: $[\tilde{y}_{t+1}, \pi_{t+1}]$ for computing expectations via Monte Carlo integration.

\subsubsection{\texttt{loss(state, expect, policy)}}
Computes the Euler equation residuals that must be minimized. This is the core of the DEQN algorithm:

\begin{align*}
\text{DIS residual:} \quad & \tilde{y}_t - \mathbb{E}_t\{\tilde{y}_{t+1}\} + \frac{1}{\sigma}(i_t - \mathbb{E}_t\{\pi_{t+1}\} - r_t^n) \\
\text{NKPC residual:} \quad & \pi_t - \beta \, \mathbb{E}_t\{\pi_{t+1}\} - \kappa \, \tilde{y}_t
\end{align*}

where $i_t = \phi_\pi \, \pi_t + \phi_y \, \tilde{y}_t + v_t$ from the Taylor rule.

Returns the mean squared residual and accuracy metrics.

\subsubsection{\texttt{get\_aggregates(simul\_policies, simul\_states)}}
Computes all economic variables from a simulation:
\begin{itemize}
    \item Output gap: $\tilde{y}_t$
    \item Inflation: $\pi_t$
    \item Nominal interest rate: $i_t = \phi_\pi \pi_t + \phi_y \tilde{y}_t + v_t$
    \item Natural rate: $r_t^n = \sigma \psi_{ya}^n (\rho_a - 1) a_t$
    \item Real interest rate: $r_t \approx i_t - \pi_t$
    \item Shocks: $a_t$, $v_t$
\end{itemize}

\subsection{DEQN Algorithm}

The Deep Equilibrium Network (DEQN) algorithm solves the model by training a neural network $\pi_\theta(s)$ to approximate the equilibrium policy function:

\begin{enumerate}
    \item \textbf{Neural Network:} Define $\pi_\theta: \mathbb{R}^2 \to \mathbb{R}^2$ mapping states $[a_t, v_t]$ to policies $[\tilde{y}_t, \pi_t]$
    
    \item \textbf{Simulate Episodes:} Generate trajectories $\{s_t, p_t\}_{t=0}^T$ using the current policy network
    
    \item \textbf{Compute Expectations:} For each state-policy pair, use Monte Carlo integration over shocks to compute $\mathbb{E}_t\{\cdot\}$ terms
    
    \item \textbf{Euler Residuals:} Evaluate how well the DIS and NKPC equations are satisfied
    
    \item \textbf{Gradient Descent:} Update network parameters $\theta$ to minimize squared Euler residuals using Adam optimizer
    
    \item \textbf{Iterate:} Repeat until convergence (accuracy $> 99\%$)
\end{enumerate}

The trained network provides a global, non-linear approximation to the model solution.

\subsection{Usage Examples}

\subsubsection{Training the Model}

\begin{lstlisting}[language=bash]
# Train the NK model (100 epochs, ~30 seconds)
python -m DEQN.econ_models.NK.train
\end{lstlisting}

\subsubsection{Running Impulse Response Analysis}

\begin{lstlisting}[language=bash]
# Compute IRFs to productivity and monetary shocks
python -m DEQN.econ_models.NK.analysis
\end{lstlisting}

This generates impulse responses over 40 quarters and saves plots showing the dynamic responses of output gap, inflation, interest rates to both productivity and monetary policy shocks.

\subsubsection{Using the Model in Python}

\begin{lstlisting}[language=Python]
from DEQN.econ_models.NK.model import Model

# Create model with default parameters
model = Model()

# Or with custom calibration
model = Model(
    beta=0.99,
    sigma=1.0,
    kappa=0.15,
    phi_pi=2.0,
    phi_y=0.5/4
)

# Access model properties
print(f"Discount factor: {model.beta}")
print(f"NKPC slope: {model.kappa}")
print(f"State dimension: {model.dim_states}")
print(f"Policy dimension: {model.dim_policies}")
\end{lstlisting}

\section{Comparison with RBC Model}

Table~\ref{tab:comparison} highlights key differences between the NK model and typical Real Business Cycle models.

\begin{table}[h]
\centering
\caption{NK Model vs. RBC Model}
\label{tab:comparison}
\begin{tabular}{@{}lll@{}}
\toprule
\textbf{Feature} & \textbf{NK Model} & \textbf{RBC Model} \\
\midrule
Prices & Sticky (Calvo) & Flexible \\
Monetary policy & Matters for real variables & Neutral \\
State variables & Exogenous shocks only & Capital + shocks \\
Endogenous states & None & Capital stock \\
Steady state policies & Zero (gap variables) & Non-zero (levels) \\
Number of FOCs & 2 (DIS + NKPC) & 1 (Euler for capital) \\
Policy instrument & Interest rate & N/A \\
Main friction & Nominal rigidities & None (RBC) or real frictions \\
\bottomrule
\end{tabular}
\end{table}

\section{References}

\begin{itemize}[leftmargin=*]
    \item \textbf{Galí, J.} (2015). \textit{Monetary Policy, Inflation, and the Business Cycle: An Introduction to the New Keynesian Framework} (2nd ed.). Princeton University Press. Chapter 3: The Basic New Keynesian Model.
    
    \item \textbf{Azinovic, M., Gaegauf, L., \& Scheidegger, S.} (2022). Deep Equilibrium Nets. \textit{International Economic Review}, 63(4), 1471--1525.
    
    \item \textbf{Woodford, M.} (2003). \textit{Interest and Prices: Foundations of a Theory of Monetary Policy}. Princeton University Press.
    
    \item \textbf{Clarida, R., Galí, J., \& Gertler, M.} (1999). The Science of Monetary Policy: A New Keynesian Perspective. \textit{Journal of Economic Literature}, 37(4), 1661--1707.
\end{itemize}

\end{document}

